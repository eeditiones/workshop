\documentclass[10pt,twoside]{article}
\usepackage{ssrq_de}  % here's all the magic...
\usepackage{hyperref}
\begin{document}
% Variablen für Kopfzeile
\def\volume{XIV. Abteilung: Die Rechtsquellen des Kantons St. Gallen, Dritter Teil: Die Landschaften und Landstädte, Band 4: Die Rechtsquellen der
     Region Werdenberg: Grafschaft Werdenberg und Herrschaft Wartau, Freiherrschaft Sax-Forstegg und Herrschaft Hohensax-Gams von Sibylle Malamud, 2019.}
\def\volid{\url{https://www.ssrq-sds-fds.ch/online/tei/SG/SSRQ_SG_III_4_235.xml}}
\linenumbers				% Zeilennummern
\normalsize					% normale Grösse, d.\,h. Quellentext (source) ist 10pt
\thispagestyle{firstpage}
\sloppy
\setcounter{subsection}{234}  % Stücknummer - 1
\article{Verzeichnis der hiesigen Geldsorten mit ihrem realen Gegenwert}
\fussy
\dating{1755 Oktober 31}

\begin{introlist}

         \item {
            Die Tabelle ist nicht datiert. Sie ist zusammen mit einer weiteren Tabelle mit
               Gattungen und \term{key000764}{Preisen} der \term{key003018}{Münzen} (\href{https://archivverzeichnis.gl.ch/home/\#/content/0d3e5bef19054808b4f9ecec4219506f}{LAGL AG III.2406:008}) als Beilage einer Kopie von
               einem Schreiben der \persname{org002906}{Drei Bünde} beigelegt, das
               am 31. Oktober 1755 an die \persname{org001189}{acht eidgenössischen Orte} geschickt worden ist.
               Die drei Bünde beschweren sich, dass ihre Landsleute in den gemeinen
               Herrschaften \placename{loc001075}{Rheintal} und \placename{loc000777}{Thurgau}, aber auch in \placename{loc000440}{Werdenberg}, \placename{loc000278}{Sax-Forstegg}
               u. a. \term{key004489}{Geld} ausgeliehen haben, das ihnen nun
               mit anderen \term{key000872}{Geldsorten} zu einem weit schlechteren
               Kurs zurückbezahlt werde (\href{https://archivverzeichnis.gl.ch/home/\#/content/e6033619e02846db817e2d60624741b7}{LAGL AG III.2406:006}).


}
         \item {Vgl. auch die Vorschläge der Eidgenössischen Kommission der Tagsatzung von 1749 über eine gemeinsame Taxation der Silber- und
               Goldsorten, die den vorliegenden \term{key004531}{Wechselkursen} sehr
               ähnlich ist (\href{                   http://permalink.snl.ch/bib/chbsg000143824}{EA, Bd. 7/2}, \href{http://digital.ub.uni-duesseldorf.de/periodical/pageview/1310151}{Art. 62, S. 75}, siehe auch das
               Münzmandat der acht Orte für das \placename{loc000925}{Sarganserland} von 1756 [\href{https://www.ssrq-sds-fds.ch/online/SG_III_2/index.html\#p_1117}{SSRQ~SG~III/2, Nr. 334}]). Die Taxationen stehen im
               Zusammenhang mit dem Versuch der regierenden acht Orte, wegen der ständigen
               \term{key000781}{Teuerung} ab Mitte des 18. Jh. einheitliche
               Umrechnungskurse in den gemeinen Herrschaften und den umliegenden Orten
               festzusetzen. Es werden wiederholt Münzmandate ausgegeben, die jedoch aufgrund
               der Uneinigkeit der einzelnen Orte zu keinem Ergebnis führen (vgl. \href{https://www.ssrq-sds-fds.ch/online/SG_III_2/index.html\#p_1117}{SSRQ SG III/2, Nr. 334, Nachbem. 1}). 


         
}
      
\end{introlist}


\begin{source} Tabelle der alhie zusezenden \term{lem013733}{geltsorten}
\vspace{1.5mm}

\noindent \term{lem004635}{Reichs müntzen}


                  1\,xr stuk
                     auf ¾\,xr
                     oder 6 hlr


                  2\,xr stuk
                     auf 1½
                       \,xr


                  3\,xr stuk
                     auf 2¼
                       \,xr


                  4\,xr stuk
                     auf 3¼
                       \,xr


                  6\,xr stuk
                     auf 5\,xr


                  12\,xr
                     stuk auf 10
                       \,xr


                  15\,xr
                     stuk auf 13½
                       \,xr


                  30\,xr stuk
                     auf 27\,xr


                  30 Monforter
                    \,xr 21½


\noindent Die \term{lem001168}{gold} und \term{lem001715}{silber sorten} betreffende könnten selbige
                     auf folgende art valutiert werden:


                  1
                     Louisblanc
                    \,2
                        8\,xr


                  1
                     ducaten
                    \, 4¼
                  


                  1
                     Mirliton
                    \, 7½


                  1
                     dublon
                    \, 7¾
                  


                  1 sonnen
                     dublon
                    \, 9½


                  1 neüen Louis
                     Dor
                    \, 9¾


                  Ein 10\,
                     stuk (die \term{lem016813.02}{Monforter} ausgenohmen als
                     welche gänzlich verbotten seyn sollen) à\, 9
                        54
                          \,xr. 


               
            
\end{source}



\manudesc{\textbf{Aufzeichnung:} LAGL AG~III.2406:007; (Doppelblatt, 1 Seite beschrieben); Papier, 22.0\,×\,36.0\,cm.}





\printnotes*
\end{document}
