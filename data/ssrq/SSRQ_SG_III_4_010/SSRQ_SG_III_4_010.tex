\documentclass[10pt,twoside]{article}
\usepackage{ssrq_de}  % here's all the magic...
\usepackage{hyperref}
\begin{document}
% Variablen für Kopfzeile
\def\volume{XIV. Abteilung: Die Rechtsquellen des Kantons St. Gallen, Dritter Teil: Die Landschaften und Landstädte, Band 4: Die Rechtsquellen der
     Region Werdenberg: Grafschaft Werdenberg und Herrschaft Wartau, Freiherrschaft Sax-Forstegg und Herrschaft Hohensax-Gams von \persname{}{Sibylle Malamud}, 2019.}
\def\volid{\url{https://www.ssrq-sds-fds.ch/online/tei/SG/SSRQ_SG_III_4_010.xml}}
\linenumbers				% Zeilennummern
\normalsize					% normale Grösse, d.\,h. Quellentext (source) ist 10pt
\thispagestyle{firstpage}
\sloppy
\setcounter{subsection}{9}  % Stücknummer - 1
\article{Friedensschluss der Grafen Albrecht~I., Albrecht~II. und Hugo~IV. von Werdenberg-Heiligenberg mit Schwyz}
\fussy
\dating{1366 Juni 29. Rheineck}
\begin{summary}

                     Die Grafen Albrecht~I., Albrecht~II. und Hugo~IV. von Werdenberg-Heiligenberg schliessen mit Schwyz Frieden wegen aller Streitsachen, insbesondere wegen Thüring und seinen Erben wegen Gefängnis, Gülten (Grundpfanddarlehen) oder anderen Forderungen.

Die Aussteller siegeln.
\end{summary}


\begin{introlist}

         \item {
            Dieser \term{key003246}{Friedensvertrag} zwischen \persname{org001191}{Schwyz} und \persname{per000269a}{Albrecht~I.}, \persname{per004996a}{Albrecht~II.} und dessen Sohn \persname{per000418a}{Hugo~IV. von Werdenberg} erscheint in keiner
               Urkundensammlung, auch nicht bei Perret (\href{http://permalink.snl.ch/bib/chbsg000110415}{UBSSG}) oder in
               den Regesten von Krüger (\href{http://permalink.snl.ch/bib/chbsg000141369}{Krüger,
                  Regesten}). Einzig in den Eidgenössischen Abschieden (\href{http://digital.ub.uni-duesseldorf.de/periodical/pageview/205767?query=Werdenberg}{EA, Bd. 1}, Art. 119) findet sich dazu ein Eintrag. Auf welche \term{key000174}{Fehde} oder Streitigkeit sich die Richtung bezieht, ist
               unklar. Laut Inhalt der Urkunde handelt es sich um eine Auseinandersetzung mit
                  \persname{org001191}{Schwyz} und besonders mit den Erben von
                  \persname{per008562a}{Thüring} um \term{key003389}{Gefangenschaft} und \term{key000058}{Schulden}. Dass sich
               Thüring und seine Erben auf \persname{per004355a}{Thüring von
                  Attinghausen}, Abt von Disentis, bezieht, der 1353 stirbt, ist unwahrscheinlich, obwohl die Werdenberger mit dem Kloster
                  \persname{org000087}{Disentis} als deren \term{key000039}{Klostervögte} wiederholt in Streit liegen (\href{http://permalink.snl.ch/bib/chbsg000141369}{Krüger, Regesten},
                  S. 186–190). Die Formulierung Thüring und seine Erben kann sich nicht auf
               den Abt beziehen, da Äbte keine Erben hinterlassen. Möglicherweise handelt es sich um
               die Familie Thüring, die 1311 mit Werner Thüring und 1342 mit einem Thüring (ohne
               Vornamen) in den Quellen als Schwyzer Ammänner erwähnt sind (\href{https://www.e-periodica.ch/digbib/view?pid=gfr-001:1888:43\#158}{Ringholz 1888,} Beilagen Nr.~XI und XXII, für den Hinweis danke
               ich Heinz Gabathuler). Da \persname{per000269a}{Albrecht~I.} zwischen
                  1352 bis 1362 in zahlreiche
               Fehden verstrickt ist (\href{http://permalink.snl.ch/bib/chbsg000150287}{Burmeister 2006}, S. 123), wird es schwierig sein, die Ursache für
               die Richtung mit Sicherheit festzulegen. Wahrscheinlich steht der Friedensvertrag in
               Zusammenhang mit den zahlreichen \term{}{Schulden} der \persname{org000019}{Werdenberg-Heiligenberger}, in welche die Familie nach
               der Belmonter und der Tosterser \term{}{Fehde} geraten ist (vgl.
               die Kommentare SSRQ~SG~III/4 9). Wegen ihrer Schulden werden die Grafen
               zwar von Kaiser \persname{per002585}{Karl~IV.} am 16. Mai 1364 aus der \term{}{Acht}
               entlassen, jedoch mit der Ergänzung, dass alle Personen, die vor einem der \term{}{Landgerichte} die Acht gegen die Grafen erlangt hatten,
               ihre Klagen am 25. Juli vor dem kaiserlichen \term{}{Hofgericht} vorbringen sollten (\href{http://permalink.snl.ch/bib/chbsg000141369}{Krüger, Regesten},
                  Nr. 399). Es ist möglich, dass \persname{org001191}{Schwyz}
               und \persname{per008562a}{Thüring} bzw. dessen Erben zu diesen \term{key004001}{Gläubigern} gehört hatten.


         
}
         \item {
            Vgl. dazu auch den Friedensschluss vom 11. November
                  1339 zwischen Graf \persname{per000269a}{Albrecht I. von
                  Werdenberg-Heiligenberg} mit den drei Waldstätten \persname{org001181}{Uri}, \persname{org001191}{Schwyz} und
                  \persname{org001192}{Unterwalden}, in dem er ihnen in seinem Gericht
               und Gebiet Frieden und Schirm zusichert (Druck: \href{http://permalink.snl.ch/bib/chbsg000135621}{UBSG}, Bd. 2,
                  Nr. 1402; \href{http://permalink.snl.ch/bib/chbsg000142708}{Mohr CD}, Bd. 2, Nr. 266; \href{http://permalink.snl.ch/bib/chbsg000071960}{Tschudi, Chronicon},
                  Bd. 4, S. 292–293; siehe auch Krüger, Regesten,
                  Nr. 281) sowie den Friedensschluss vom 29. November
                  1339 zwischen Uri, Schwyz und Unterwalden einerseits und \persname{per004355}{Thüring von Attinghausen}, Abt von Disentis, Graf
                  \persname{per000269a}{Albrecht I. von Werdenberg-Heiligenberg} und
               seinen Verbündeten andererseits (Edition: \href{http://permalink.snl.ch/bib/chbsg000102701}{QW I}, Bd. 3.1,
                  Nr. 293).


         
}
         \item {
            Die Urkunde belegt, dass \persname{per000269a}{Albrecht I. von
      Werdenberg-Heiligenberg} im Juni 1366 noch am Leben ist. Bisher galt
     die Urkunde von 1364 als Letzterwähnung (\href{http://permalink.snl.ch/bib/chbsg000141369}{Krüger, Regesten},
     Nr. 399). Er muss jedoch bald nach diesem Friedensschluss gestorben sein, mit
     Sicherheit vor dem 1. Oktober 1367 (\href{http://permalink.snl.ch/bib/chbsg000141369}{Krüger, Regesten},
      Nr. 403).


         
}
      
\end{introlist}


\begin{source}

               
                  
                  
                     
                        Wir, graf \persname{per000269a}{Albrecht von Werdenberg}
                        der alt, graf \persname{per004996a}{Albrecht der jung} und
                           \leavevmode\textnotestart{a}{Unsichere Lesung.}des sun\textnoteend{a}, graf \persname{per000418a}{Hug}, knden und vergechen offenlich mit urknd dis briefs,
                        fr uns und nser erben, daz wir frntlichen und lieplichen verricht
                        sin umb alle die stss und missehellung, ds si mit briefen oder mit
                        anderen dingen, so wir untz uf disen hütigen tag gehept
                        hant, mit den von \persname{org001191}{Switz} und irem land
                        gemeinlich und sunderlich mit \persname{per008562a}{Thiring} und sinen erben, es si von \term{lem001256}{vanunst} wegen oder von \term{lem000340}{glt} wegen
                        und von aller der \term{lem002027}{ansprach} wegen, so si
                        z ns und wir z inen hatten ann alle geverde.



                     Und darumb, daz dis \term{lem005912}{richtung} steif und
                        war und unverkert belibe, so henken wir, die vorgeschriben herren
                           \persname{org000019}{von Werdenberg}, alle drüe nser aygen
                        yngesigel an disen offennen gegenwirtigen brief, ns und nsern
                        erbenn ze einer vergicht der sache. Der gebenn ist ze Rinegg, an sant
                              \persname{per000438}{Peter}s tag nach \persname{per000353}{Cristus} gebrdt dr cechenn hundert
                           jar und darnach inn sechs und sechtzigostem jare.



                  


               
               
               
                  \medskip
\noindent
{\small \textit{[Vermerk auf der Rückseite von Hand des 15.~Jh.:]} Ein richtung brieff zwüschent
                     graff \persname{per000269a}{Albrecht von Werdenberg}
                     und anderen herren harinne genempt \leavevmode\textnote[b]{Hinzufügung unterhalb der Zeile von späterer Hand: \textup{und denen von \persname{org001191}{Schweytz} 1366}.}}



               
               
                  \medskip
\noindent
{\small \textit{[Registraturvermerk unterhalb des Textes:]} N\textsuperscript{o} 11}



               
            
\end{source}



\manudesc{\textbf{Original:} StASZ HA.II.190; Pergament, 23.0\,×\,16.0\,cm (Plica: 4.0\,cm).}




\manudesc{\textbf{URL:} \url{https://query.staatsarchiv.sz.ch/detail.aspx?id=369445}}



\printnotes*
\end{document}
