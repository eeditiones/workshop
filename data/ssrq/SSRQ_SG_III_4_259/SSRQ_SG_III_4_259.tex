\documentclass[10pt,twoside]{article}
\usepackage{ssrq_de}  % here's all the magic...
\usepackage{hyperref}
\begin{document}
% Variablen für Kopfzeile
\def\volume{XIV. Abteilung: Die Rechtsquellen des Kantons St. Gallen, Dritter Teil: Die Landschaften und Landstädte, Band 4: Die Rechtsquellen der
     Region Werdenberg: Grafschaft Werdenberg und Herrschaft Wartau, Freiherrschaft Sax-Forstegg und Herrschaft Hohensax-Gams von Sibylle Malamud, 2019.}
\def\volid{\url{https://www.ssrq-sds-fds.ch/online/tei/SG/SSRQ_SG_III_4_259.xml}}
\linenumbers				% Zeilennummern
\normalsize					% normale Grösse, d.\,h. Quellentext (source) ist 10pt
\thispagestyle{firstpage}
\sloppy
\setcounter{subsection}{258}  % Stücknummer - 1
\article{Schwyz entlässt Gams in die Freiheit}
\fussy
\dating{1798 März 24}
\begin{summary}
Landammann und Landrat von Schwyz urkunden, dass, nachdem die Landvogtei Gaster in die Freiheit
                        entlassen wurde, auch Gams in die Freiheit entlassen wird, jedoch mit dem
                        Vorbehalt, dass Gams die katholische Religion beibehält, das Privat- und
                        Staatseigentum sicher bleibt und die alljährlichen Zinsen laut Zinsbrief und
                        Zusagen vom 21. März 1798 durch die Gamser Abgeordneten, Säckelmeister
                        Johann Hardegger und Michael Hardegger, bis zur Auslösung entrichtet werden.
                        Auch im Fall eines Auszugs in den Krieg soll keiner den anderen mit Kosten
                        beladen und sich gegenseitig nicht mit neuen Zöllen und Weggeldern
                        beschweren.

 Der Aussteller siegelt mit dem
   Sekretsiegel.


\end{summary}


\begin{introlist}

         \item {
            Die vorliegende Erklärung von \persname{org003265}{Schwyz} zur Unabhängigkeit
               von \placename{loc006040}{Hohensax-Gams} ist flüchtig geschrieben, schwer
               lesbar und enthält viele Streichungen, weshalb es sich um einen Entwurf handeln muss. Eine
               gesiegelte Originalurkunde ist nicht auffindbar.
               


}
         \item {
            Über die geschichtlichen Ereignisse in der Herrschaft \placename{loc006040}{Hohensax-Gams} während des Übergangs zur Helvetischen Republik ist
               wenig bekannt. Die Darstellungen in der Literatur (\href{http://permalink.snl.ch/bib/chbsg000081018}{Kessler 1985},
                  S. 56–59; \href{http://permalink.snl.ch/bib/chbsg000141030}{Reich
                     1998}, S. 46–47) beruhen weitgehend auf den Ausführungen von
                     \href{http://permalink.snl.ch/bib/chbsg000113555}{Senn, Chronik},
                     S. 331, die ohne Quellenangaben und grösstenteils von
               zeitgenössischen Erzählungen stammen. \persname{org001881}{Gams} hat
               sich jedoch den Freiheitsbewegungen in der unmittelbaren Nachbarschaft nicht
               angeschlossen: Am 7. März 1798 versichert Gams in
               einem Schreiben dem Stand \persname{org001191}{Schwyz}, dass sie keinen
               Freiheitsbaum aufgerichtet hätten, dass es \textup{weit entfernet seye von unß, dies
                  unordentliche beyspille nachzuahmen, die wir ihrer stifftung und ursprungs wegen
                  für verabscheuchungswürdig ansehen und einem getreüen \term{lem000900}{volck}, daß seiner rechtmäßigen \term{lem000720}{obrigkeit}
                  ganz ergeben}, vollkommen widerspreche. Vielmehr hätten sie sich den
               Aufforderungen ihrer Nachbarn, solche Bäume aufzurichten, widersetzt und hätten sogar
                  \term{key000580}{Wachen} aufgestellt, damit ihnen niemand
               Freiheitsbäume aufzwingen könne. An der heutigen \term{key004360}{Landsgemeinde} hätten sie zudem beschlossen, ihren \term{key000817}{Vertrag} von 1497 (SSRQ~SG~III/4 94;
               im Schreiben versehentlich 1479) mit den beiden Orten \persname{org001191}{Schwyz} und \persname{org001197}{Glarus} erhalten und
               schützen zu wollen (StASZ
               HA.IV.470.003, Nr. 64. Auch der Landvogt von Sax-Forstegg, der am 6. Februar über Unruhen in den Nachbargebieten berichtet, erwähnt nur \placename{loc000440}{Werdenberg}, das \placename{loc001075}{Rheintal} und das \placename{loc000782}{Toggenburg}, nicht aber Hohensax-Gams [StAZH A 93.3, Nr. 158]).


            Am 10. März 1798 wird an einer ausserordentlichen
               Landsgemeinde in Schwyz beschlossen, dass alle Angehörigen der Landschaften, die noch nicht
               ausdrücklich in die Freiheit entlassen worden seien, von heute an für frei erklärt sein sollen
               (\href{https://query.staatsarchiv.sz.ch/detail.aspx?ID=371412}{StASZ
                  HA.III.285}, S. 500 [Pdf, S. 176]; Druck: \href{http://permalink.snl.ch/bib/chbsg000091380}{Wiget 1997}, S. 46).
               Doch erst als laut Inhalt des vorliegenden Entwurfs am 21. März
                  1798 die beiden Gamser Abgeordneten Säckelmeister \persname{per010783}{Johann
                     Hardegger} und \persname{per010784}{Michael Hardegger} versichern,
               dass die jährlichen \term{key000154}{Zinsen} bis zur \term{key000685}{Ablösung} bezahlt würden, entlässt Schwyz auch die \persname{org001881}{Gamser} in die Freiheit. Glarus hatte Gams bereits am 11. März 1798 für frei und unabhängig erklärt mit der Bedingung, dass
               sie Schwyz und Glarus die \textup{gült brief wie bis anhin verzinset oder das capital bezalt
                  hat} (\href{https://archivverzeichnis.gl.ch/home/\#/content/22acebab9be045be9a31e1564d611365}{LAGL AAA 1/87 S. 429}).


            Am 12. Mai 1798 erscheinen \persname{per010785}{Anton Lenherr} und \persname{per010784}{Michael
                  Hardegger} als Abgeordnete der Gemeinde \persname{org001881}{Gams} vor General \persname{per010790a}{von Schauenburg}
               und zeigen an, dass sie die Helvetische Konstitution einstimmig angenommen haben.
               Dieser rät ihnen, die Annahme dem Direktor der \placename{loc008548.01}{Helvetischen Republik} in \placename{loc000121}{Aarau}
               anzuzeigen (OGA Gams Nr. 212). Nach Senn verlangen Schwyz und Glarus 1804 den 1497 vorgeschossenen Kaufbetrag von 4920 Gulden (laut Zinsbrief von 1497 sind es allerdings 4000 Gulden, siehe SSRQ SG III/4 93, Kommentar 2), für den Gams jährlich
               über Jahrhunderte 200 Gulden Zins bezahlt hat, zurück. Während Schwyz ihr Kapital von
               1750 Gulden der Pfarrkirche \placename{loc006041}{Gams} übergibt,
               behält Glarus seine gesamte Einlage (\href{http://permalink.snl.ch/bib/chbsg000113555}{Senn, Chronik},
                  S. 103; \href{http://permalink.snl.ch/bib/chbsg000081018}{Kessler 1985}, S. 39–42).


            Die Gamser erhalten zwar im Vergleich zu ihren Nachbarn erst spät ihre Unabhängigkeit; der
               Übergang erfolgt jedoch ohne grössere Unruhen. Naheliegend ist dabei die Vermutung von Kessler,
               die Zurückhaltung der Gamser in Sachen Unabhängigkeit mit der Tatsache in Verbindung zu
               bringen, dass die Gamser seit dem 15. Jh. mehr
               Freiheiten besassen als die umliegenden \term{key000649}{Herrschaftsgebiete}
               (SSRQ~SG~III/4 59; SSRQ~SG~III/4 94; \href{http://permalink.snl.ch/bib/chbsg000081018}{Kessler 1985},
                  S. 56–59). 


         
}
         \item {Die gemeine Landvogtei \placename{loc000842}{Gaster}, der Hohensax-Gams
               verwaltungstechnisch angegliedert ist, wird bereits am 6.
                  März 1798 durch \persname{org001191}{Schwyz} aus dem
               Untertanenverhältnis entlassen (Druck: \href{https://www.ssrq-sds-fds.ch/online/SG_III_1/index.html\#p_241}{SSRQ SG III/1, Nr. 146a}; \href{                         http://permalink.snl.ch/bib/chbsg000143823}{EA,
                        Bd. 8}, \href{http://digital.ub.uni-duesseldorf.de/periodical/pageview/1387096}{S. 674}), gefolgt von \persname{org001197}{Glarus} am 11. März 1798 (Druck:
                     \href{https://www.ssrq-sds-fds.ch/online/GL_1.1/index.html\#p_519}{SSRQ GL/1.1, Nr. 192F}; \href{https://www.ssrq-sds-fds.ch/online/SG_III_1/index.html\#p_242}{SSRQ SG III/1, Nr. 146b}). Hohensax-Gams wird darin
               nicht erwähnt. 


}
      
\end{introlist}


\begin{source}

               
                  
                     Wir, \persname{org003265}{landamman und gesessner landrath zu \placename{loc000731.02}{Schweiz}}, urkunden für unser ort anmit, daß
                     wir zufolg der lesten mayen landsgemeinde\leavevmode\textnote[a]{Streichung: \textup{erkantnuss}.} und die
                     unter 8\textsuperscript{ten} märz
                     von einem drey\leavevmode\textnote[b]{Korrigiert aus: \textup{drey drey}.}fachen landrath in krafft einer landsgemeind bereits schon ausgefälten
                     erkantniß, die\leavevmode\textnote[c]{Streichung: \textup{zu}.}
                     der landvogtey \placename{loc000842}{Gaster}\leavevmode\textnote[d]{Streichung: \textup{zugehö}.} anhängig gewesene gemeind \placename{loc000211}{Gambs}\leavevmode\textnote[e]{Streichung: \textup{so}.} von nun an je und zu allenzeiten als \term{lem014460}{frey}
                     und ohnabhängig erklären und\leavevmode\textnote[f]{Streichung: \textup{erklären}.} erkennen, jedoch mit
                     dem deütlichen vorbehalt, daß in gemelter gemeind ihre alte catholische
                     \term{lem010186}{relligion}
                     beybehalten, daß \term{lem016789.02}{privat} und \term{lem016790.02}{staats eigenthum} gesichert und, laut\leavevmode\textnote[g]{Streichung mit Textverlust (2 Wörter).} der zusagen von denen unterm 21.ten diß mit vollmacht abgeordneten hh \term{lem001645}{sekelmeister}
                     \persname{per010783a}{Johann Hardegger} und \persname{per010784}{Michel Hardegger}, von dem \term{lem002384}{zinßbrief} biß zu deßen
                     \term{lem014312}{auslosung}
                     der alljährliche
                     \term{lem000237}{zinß} wie bishero entrichtet und bezalt, auch im
                     fall eines \term{lem007247}{auszugs} kein theil dem andren mit kösten
                     beladen und inskünftig wir wechselseittig einander\leavevmode\textnote[h]{Streichung: \textup{mit}.}  weder mit neuen
                     \term{lem000349}{zollen} noch \term{lem001110}{weggeldern}
                     beschwehren und so auch ermelter gemeind \noindent \pb{} \textit{[fol.~1v]}
                     die aus ih[...]tende\leavevmode\textnote[i]{Unsichere Lesung.}
                     \term{lem002384}{zinsbriefen}
                     zu lasen gestatten seyn solle.


                        In urkund, wessen wir disere \term{lem003204}{befreyung} mit unser
                        stands sekret insigill verwahret haben ausfertigen laßen, geben, den
                        24. märz 1798.


                           [Locus sigilli]
                           \persname{}{JAV}, landschreiber, manu propria


               
               
            
\end{source}



\manudesc{\textbf{Entwurf:} StASZ HA.IV. 470.004, Nr.~115; (Einzelblatt); J A V, Landschreiber; Papier, 20.5\,×\,33.0\,cm.}





\printnotes*
\end{document}
